\nonstopmode{}
\documentclass[a4paper]{book}
\usepackage[times,inconsolata,hyper]{Rd}
\usepackage{makeidx}
\usepackage[utf8]{inputenc} % @SET ENCODING@
% \usepackage{graphicx} % @USE GRAPHICX@
\makeindex{}
\begin{document}
\chapter*{}
\begin{center}
{\textbf{\huge Package `ConceptSetDiagnostics'}}
\par\bigskip{\large \today}
\end{center}
\inputencoding{utf8}
\ifthenelse{\boolean{Rd@use@hyper}}{\hypersetup{pdftitle = {ConceptSetDiagnostics: Concept Set Diagnostics}}}{}
\begin{description}
\raggedright{}
\item[Type]\AsIs{Package}
\item[Title]\AsIs{Concept Set Diagnostics}
\item[Version]\AsIs{0.0.2}
\item[Author]\AsIs{Gowtham Rao [aut, cre]}
\item[Date]\AsIs{2022-07-24}
\item[Maintainer]\AsIs{Gowtham Rao }\email{rao@ohdsi.org}\AsIs{}
\item[Description]\AsIs{A package to work with OMOP concepts and cohort concept sets.}
\item[Depends]\AsIs{DatabaseConnector (>= 5.0.0),
dplyr,
R (>= 4.0.0)}
\item[Imports]\AsIs{checkmate,
CirceR,
purrr,
RJSONIO,
rlang,
scales,
SqlRender,
stringr,
stringdist,
tidyr,
tidyselect}
\item[Suggests]\AsIs{readr,
remotes,
rmarkdown,
knitr,
testthat,
withr}
\item[Remotes]\AsIs{ohdsi/CirceR,
ohdsi/SqlRender}
\item[License]\AsIs{Apache License (>= 2)}
\item[RoxygenNote]\AsIs{7.2.0}
\item[Encoding]\AsIs{UTF-8}
\item[Language]\AsIs{en-US}
\end{description}
\Rdcontents{\R{} topics documented:}
\inputencoding{utf8}
\HeaderA{convertConceptSetDataFrameToExpression}{Convert concept set expression in a data frame format convert to R (list) expression}{convertConceptSetDataFrameToExpression}
%
\begin{Description}\relax
Convert concept set expression in a data frame format convert to R (list) expression
\end{Description}
%
\begin{Usage}
\begin{verbatim}
convertConceptSetDataFrameToExpression(
  conceptSetExpressionDataFrame,
  selectAllDescendants = FALSE,
  updateVocabularyFields = FALSE,
  connectionDetails = NULL,
  connection = NULL,
  vocabularyDatabaseSchema = NULL
)
\end{verbatim}
\end{Usage}
%
\begin{Arguments}
\begin{ldescription}
\item[\code{conceptSetExpressionDataFrame}] Concept set expression in data frame format with required fields
conceptId. If includeMapped, isExcluded or includeDescendants
are missing value or is not existent - it is assumed to be FALSE.
All column names should be in camelCase format.

\item[\code{selectAllDescendants}] Do you want to over ride the concept set
expression by add select descendants for concept ids
in concept set expression.

\item[\code{updateVocabularyFields}] Do you want to update the details about concepts from the vocabulary tables such as domain, 
vocabulary, concept name? If yes, then connection or connectionDetails to a 
remote db with OMOP vocabulary tables is needed.

\item[\code{connectionDetails}] An object of type \code{connectionDetails} as created using the
\code{\LinkA{createConnectionDetails}{createConnectionDetails}} function in the
DatabaseConnector package. Can be left NULL if \code{connection} is
provided.

\item[\code{connection}] An object of type \code{connection} as created using the
\code{\LinkA{connect}{connect}} function in the
DatabaseConnector package. Can be left NULL if \code{connectionDetails}
is provided, in which case a new connection will be opened at the start
of the function, and closed when the function finishes.

\item[\code{vocabularyDatabaseSchema}] The schema name of containing the vocabulary tables.
\end{ldescription}
\end{Arguments}
%
\begin{Value}
Returns a R list object
\end{Value}
\inputencoding{utf8}
\HeaderA{convertConceptSetExpressionToDataFrame}{convert a concept set expression R list object into a data frame object}{convertConceptSetExpressionToDataFrame}
%
\begin{Description}\relax
convert a concept set expression R list object into a data frame object
\end{Description}
%
\begin{Usage}
\begin{verbatim}
convertConceptSetExpressionToDataFrame(
  conceptSetExpression,
  updateVocabularyFields = FALSE,
  connection = NULL,
  connectionDetails = NULL,
  tempEmulationSchema = getOption("sqlRenderTempEmulationSchema"),
  vocabularyDatabaseSchema = NULL
)
\end{verbatim}
\end{Usage}
%
\begin{Arguments}
\begin{ldescription}
\item[\code{conceptSetExpression}] An R-object (list) with expression of the concept set.

\item[\code{updateVocabularyFields}] Do you want to update the details about concepts from the vocabulary tables such as domain, 
vocabulary, concept name? If yes, then connection or connectionDetails to a 
remote db with OMOP vocabulary tables is needed.

\item[\code{connection}] An object of type \code{connection} as created using the
\code{\LinkA{connect}{connect}} function in the
DatabaseConnector package. Can be left NULL if \code{connectionDetails}
is provided, in which case a new connection will be opened at the start
of the function, and closed when the function finishes.

\item[\code{connectionDetails}] An object of type \code{connectionDetails} as created using the
\code{\LinkA{createConnectionDetails}{createConnectionDetails}} function in the
DatabaseConnector package. Can be left NULL if \code{connection} is
provided.

\item[\code{tempEmulationSchema}] Some database platforms like Oracle and Impala do not truly support temp tables. To emulate temp 
tables, provide a schema with write privileges where temp tables can be created.

\item[\code{vocabularyDatabaseSchema}] The schema name of containing the vocabulary tables.
\end{ldescription}
\end{Arguments}
%
\begin{Value}
Returns a tibble data frame.
\end{Value}
\inputencoding{utf8}
\HeaderA{extractConceptSetsInCohortDefinition}{Extract concept set expressions from cohort definition expression.}{extractConceptSetsInCohortDefinition}
%
\begin{Description}\relax
Given a cohort expression, this function extracts the concept set
expressions from cohort definition expression.
\end{Description}
%
\begin{Usage}
\begin{verbatim}
extractConceptSetsInCohortDefinition(cohortExpression)
\end{verbatim}
\end{Usage}
%
\begin{Arguments}
\begin{ldescription}
\item[\code{cohortExpression}] A R-object (list) that represents cohort definition expression. This is derived from cohort expression json 
using RJSONIO::fromJSON(content = json, digits = 23). Note: it is important to use digits = 23, otherwise
numerical precision may be lost for large integer values like conceptId's in cohort definition. The cohort
expression JSON is commonly generated using OHDSI tools like Atlas or CapR.
\end{ldescription}
\end{Arguments}
%
\begin{Value}
Returns a tibble data frame.
\end{Value}
\inputencoding{utf8}
\HeaderA{extractConceptSetsInCohortDefinitionSet}{Extract concept sets from cohort definition set}{extractConceptSetsInCohortDefinitionSet}
%
\begin{Description}\relax
given a cohort definition set (data frame with cohortId, json), this function
extracts the concept set json and sql for all cohorts,
compares concept sets across cohort definitions, assigns unique id.
\end{Description}
%
\begin{Usage}
\begin{verbatim}
extractConceptSetsInCohortDefinitionSet(cohortDefinitionSet)
\end{verbatim}
\end{Usage}
%
\begin{Arguments}
\begin{ldescription}
\item[\code{cohortDefinitionSet}] The \code{cohortDefinitionSet} argument must be a data frame 
with at least the following columns. 
\begin{description}

\item[cohortId] The cohort Id is the id used to identify  a 
cohort definition. This is required to be unique. 
It is usually used to create file names.
\item[cohortName] The full name of the cohort.
\item[json] The JSON cohort definition for the cohort.
\item[sql] The SQL of the cohort definition rendered from the cohort json.

\end{description}

\end{ldescription}
\end{Arguments}
%
\begin{Value}
Returns a tibble data frame.
\end{Value}
\inputencoding{utf8}
\HeaderA{findOrphanConcepts}{Get all the domain id(s) in the vocabulary schema.}{findOrphanConcepts}
%
\begin{Description}\relax
Get all the domain id(s) in the vocabulary schema.

Find orphan concepts for a concept set expression.
\end{Description}
%
\begin{Usage}
\begin{verbatim}
findOrphanConcepts(
  conceptSetExpression,
  vocabularyDatabaseSchema = "vocabulary",
  connection = NULL,
  connectionDetails = NULL,
  tempEmulationSchema = getOption("sqlRenderTempEmulationSchema")
)

findOrphanConcepts(
  conceptSetExpression,
  vocabularyDatabaseSchema = "vocabulary",
  connection = NULL,
  connectionDetails = NULL,
  tempEmulationSchema = getOption("sqlRenderTempEmulationSchema")
)
\end{verbatim}
\end{Usage}
%
\begin{Arguments}
\begin{ldescription}
\item[\code{vocabularyDatabaseSchema}] The schema name of containing the vocabulary tables.

\item[\code{connection}] An object of type \code{connection} as created using the
\code{\LinkA{connect}{connect}} function in the
DatabaseConnector package. Can be left NULL if \code{connectionDetails}
is provided, in which case a new connection will be opened at the start
of the function, and closed when the function finishes.

\item[\code{connectionDetails}] An object of type \code{connectionDetails} as created using the
\code{\LinkA{createConnectionDetails}{createConnectionDetails}} function in the
DatabaseConnector package. Can be left NULL if \code{connection} is
provided.

\item[\code{tempEmulationSchema}] Some database platforms like Oracle and Impala do not truly support temp tables. To emulate temp 
tables, provide a schema with write privileges where temp tables can be created.

\item[\code{conceptIds}] An array of Concept ids.
\end{ldescription}
\end{Arguments}
%
\begin{Value}
Returns a tibble data frame.

Returns a tibble data frame.
\end{Value}
\inputencoding{utf8}
\HeaderA{getConceptAncestor}{get concept ancestor}{getConceptAncestor}
%
\begin{Description}\relax
given an array of conceptIds, get their ancestor and descendants.
\end{Description}
%
\begin{Usage}
\begin{verbatim}
getConceptAncestor(
  conceptIds,
  connection = NULL,
  connectionDetails = NULL,
  tempEmulationSchema = getOption("sqlRenderTempEmulationSchema"),
  vocabularyDatabaseSchema = "vocabulary"
)
\end{verbatim}
\end{Usage}
%
\begin{Arguments}
\begin{ldescription}
\item[\code{conceptIds}] An array of Concept ids.

\item[\code{connection}] An object of type \code{connection} as created using the
\code{\LinkA{connect}{connect}} function in the
DatabaseConnector package. Can be left NULL if \code{connectionDetails}
is provided, in which case a new connection will be opened at the start
of the function, and closed when the function finishes.

\item[\code{connectionDetails}] An object of type \code{connectionDetails} as created using the
\code{\LinkA{createConnectionDetails}{createConnectionDetails}} function in the
DatabaseConnector package. Can be left NULL if \code{connection} is
provided.

\item[\code{tempEmulationSchema}] Some database platforms like Oracle and Impala do not truly support temp tables. To emulate temp 
tables, provide a schema with write privileges where temp tables can be created.

\item[\code{vocabularyDatabaseSchema}] The schema name of containing the vocabulary tables.
\end{ldescription}
\end{Arguments}
%
\begin{Value}
Returns a tibble data frame.
\end{Value}
\inputencoding{utf8}
\HeaderA{getConceptDescendant}{get concept descendant}{getConceptDescendant}
%
\begin{Description}\relax
given an array of conceptIds, get their descendants.
\end{Description}
%
\begin{Usage}
\begin{verbatim}
getConceptDescendant(
  conceptIds,
  connection = NULL,
  connectionDetails = NULL,
  tempEmulationSchema = getOption("sqlRenderTempEmulationSchema"),
  vocabularyDatabaseSchema = "vocabulary"
)
\end{verbatim}
\end{Usage}
%
\begin{Arguments}
\begin{ldescription}
\item[\code{conceptIds}] An array of Concept ids.

\item[\code{connection}] An object of type \code{connection} as created using the
\code{\LinkA{connect}{connect}} function in the
DatabaseConnector package. Can be left NULL if \code{connectionDetails}
is provided, in which case a new connection will be opened at the start
of the function, and closed when the function finishes.

\item[\code{connectionDetails}] An object of type \code{connectionDetails} as created using the
\code{\LinkA{createConnectionDetails}{createConnectionDetails}} function in the
DatabaseConnector package. Can be left NULL if \code{connection} is
provided.

\item[\code{tempEmulationSchema}] Some database platforms like Oracle and Impala do not truly support temp tables. To emulate temp 
tables, provide a schema with write privileges where temp tables can be created.

\item[\code{vocabularyDatabaseSchema}] The schema name of containing the vocabulary tables.
\end{ldescription}
\end{Arguments}
%
\begin{Value}
Returns a tibble data frame.
\end{Value}
\inputencoding{utf8}
\HeaderA{getConceptIdDetails}{get concept id details}{getConceptIdDetails}
%
\begin{Description}\relax
given an array of conceptIds, get their details
\end{Description}
%
\begin{Usage}
\begin{verbatim}
getConceptIdDetails(
  conceptIds,
  connection = NULL,
  connectionDetails = NULL,
  vocabularyDatabaseSchema = "vocabulary",
  tempEmulationSchema = getOption("sqlRenderTempEmulationSchema")
)
\end{verbatim}
\end{Usage}
%
\begin{Arguments}
\begin{ldescription}
\item[\code{conceptIds}] An array of Concept ids.

\item[\code{connection}] An object of type \code{connection} as created using the
\code{\LinkA{connect}{connect}} function in the
DatabaseConnector package. Can be left NULL if \code{connectionDetails}
is provided, in which case a new connection will be opened at the start
of the function, and closed when the function finishes.

\item[\code{connectionDetails}] An object of type \code{connectionDetails} as created using the
\code{\LinkA{createConnectionDetails}{createConnectionDetails}} function in the
DatabaseConnector package. Can be left NULL if \code{connection} is
provided.

\item[\code{vocabularyDatabaseSchema}] The schema name of containing the vocabulary tables.

\item[\code{tempEmulationSchema}] Some database platforms like Oracle and Impala do not truly support temp tables. To emulate temp 
tables, provide a schema with write privileges where temp tables can be created.
\end{ldescription}
\end{Arguments}
%
\begin{Value}
Returns a tibble data frame.
\end{Value}
\inputencoding{utf8}
\HeaderA{getConceptPrevalenceCounts}{get concept id count}{getConceptPrevalenceCounts}
%
\begin{Description}\relax
Get the count for an array of concept id(s) from concept prevalence table.
\end{Description}
%
\begin{Usage}
\begin{verbatim}
getConceptPrevalenceCounts(
  conceptIds,
  connection = NULL,
  connectionDetails = NULL,
  conceptPrevalenceSchema,
  tempEmulationSchema = getOption("sqlRenderTempEmulationSchema")
)
\end{verbatim}
\end{Usage}
%
\begin{Arguments}
\begin{ldescription}
\item[\code{conceptIds}] An array of Concept ids.

\item[\code{connection}] An object of type \code{connection} as created using the
\code{\LinkA{connect}{connect}} function in the
DatabaseConnector package. Can be left NULL if \code{connectionDetails}
is provided, in which case a new connection will be opened at the start
of the function, and closed when the function finishes.

\item[\code{connectionDetails}] An object of type \code{connectionDetails} as created using the
\code{\LinkA{createConnectionDetails}{createConnectionDetails}} function in the
DatabaseConnector package. Can be left NULL if \code{connection} is
provided.

\item[\code{conceptPrevalenceSchema}] The schema name that has the concept prevalence table. The following
tables are expected to be present. recommender\_set,
cp\_master, recommended\_blacklist.

\item[\code{tempEmulationSchema}] Some database platforms like Oracle and Impala do not truly support temp tables. To emulate temp 
tables, provide a schema with write privileges where temp tables can be created.
\end{ldescription}
\end{Arguments}
%
\begin{Value}
Returns a tibble data frame.
\end{Value}
\inputencoding{utf8}
\HeaderA{getConceptRecordCount}{Given conceptId(s) get the record count.}{getConceptRecordCount}
%
\begin{Description}\relax
Given conceptId(s) get the record count.
\end{Description}
%
\begin{Usage}
\begin{verbatim}
getConceptRecordCount(
  conceptIds,
  connection = NULL,
  connectionDetails = NULL,
  cdmDatabaseSchema,
  vocabularyDatabaseSchema = cdmDatabaseSchema,
  tempEmulationSchema = getOption("sqlRenderTempEmulationSchema"),
  minCellCount = 0
)
\end{verbatim}
\end{Usage}
%
\begin{Arguments}
\begin{ldescription}
\item[\code{conceptIds}] An array of Concept ids.

\item[\code{connection}] An object of type \code{connection} as created using the
\code{\LinkA{connect}{connect}} function in the
DatabaseConnector package. Can be left NULL if \code{connectionDetails}
is provided, in which case a new connection will be opened at the start
of the function, and closed when the function finishes.

\item[\code{connectionDetails}] An object of type \code{connectionDetails} as created using the
\code{\LinkA{createConnectionDetails}{createConnectionDetails}} function in the
DatabaseConnector package. Can be left NULL if \code{connection} is
provided.

\item[\code{cdmDatabaseSchema}] Schema name where your patient-level data in OMOP CDM format resides.
Note that for SQL Server, this should include both the database and
schema name, for example 'cdm\_data.dbo'.

\item[\code{vocabularyDatabaseSchema}] The schema name of containing the vocabulary tables.

\item[\code{tempEmulationSchema}] Some database platforms like Oracle and Impala do not truly support temp tables. To emulate temp 
tables, provide a schema with write privileges where temp tables can be created.

\item[\code{minCellCount}] The minimum cell count for fields containing person/subject count.
\end{ldescription}
\end{Arguments}
%
\begin{Value}
Returns a tibble data frame.
\end{Value}
\inputencoding{utf8}
\HeaderA{getConceptRelationship}{given a list of conceptIds, get their relationship}{getConceptRelationship}
%
\begin{Description}\relax
given a list of conceptIds, get their relationship
\end{Description}
%
\begin{Usage}
\begin{verbatim}
getConceptRelationship(
  conceptIds,
  connection = NULL,
  connectionDetails = NULL,
  tempEmulationSchema = getOption("sqlRenderTempEmulationSchema"),
  vocabularyDatabaseSchema = "vocabulary"
)
\end{verbatim}
\end{Usage}
%
\begin{Arguments}
\begin{ldescription}
\item[\code{conceptIds}] An array of Concept ids.

\item[\code{connection}] An object of type \code{connection} as created using the
\code{\LinkA{connect}{connect}} function in the
DatabaseConnector package. Can be left NULL if \code{connectionDetails}
is provided, in which case a new connection will be opened at the start
of the function, and closed when the function finishes.

\item[\code{connectionDetails}] An object of type \code{connectionDetails} as created using the
\code{\LinkA{createConnectionDetails}{createConnectionDetails}} function in the
DatabaseConnector package. Can be left NULL if \code{connection} is
provided.

\item[\code{tempEmulationSchema}] Some database platforms like Oracle and Impala do not truly support temp tables. To emulate temp 
tables, provide a schema with write privileges where temp tables can be created.

\item[\code{vocabularyDatabaseSchema}] The schema name of containing the vocabulary tables.
\end{ldescription}
\end{Arguments}
%
\begin{Value}
Returns a tibble data frame.
\end{Value}
\inputencoding{utf8}
\HeaderA{getConceptSynonym}{given a list of conceptIds, get their synonyms}{getConceptSynonym}
%
\begin{Description}\relax
given a list of conceptIds, get their synonyms
\end{Description}
%
\begin{Usage}
\begin{verbatim}
getConceptSynonym(
  conceptIds,
  connection = NULL,
  connectionDetails = NULL,
  tempEmulationSchema = getOption("sqlRenderTempEmulationSchema"),
  vocabularyDatabaseSchema = "vocabulary"
)
\end{verbatim}
\end{Usage}
%
\begin{Arguments}
\begin{ldescription}
\item[\code{conceptIds}] An array of Concept ids.

\item[\code{connection}] An object of type \code{connection} as created using the
\code{\LinkA{connect}{connect}} function in the
DatabaseConnector package. Can be left NULL if \code{connectionDetails}
is provided, in which case a new connection will be opened at the start
of the function, and closed when the function finishes.

\item[\code{connectionDetails}] An object of type \code{connectionDetails} as created using the
\code{\LinkA{createConnectionDetails}{createConnectionDetails}} function in the
DatabaseConnector package. Can be left NULL if \code{connection} is
provided.

\item[\code{tempEmulationSchema}] Some database platforms like Oracle and Impala do not truly support temp tables. To emulate temp 
tables, provide a schema with write privileges where temp tables can be created.

\item[\code{vocabularyDatabaseSchema}] The schema name of containing the vocabulary tables.
\end{ldescription}
\end{Arguments}
\inputencoding{utf8}
\HeaderA{getCountOfSourceCodesMappedToStandardConcept}{Given conceptId(s) get the counts of occurrence with mapping.}{getCountOfSourceCodesMappedToStandardConcept}
%
\begin{Description}\relax
Given conceptId(s) get the counts of occurrence with mapping.
\end{Description}
%
\begin{Usage}
\begin{verbatim}
getCountOfSourceCodesMappedToStandardConcept(
  conceptIds,
  connection = NULL,
  connectionDetails = NULL,
  cdmDatabaseSchema,
  tempEmulationSchema = getOption("sqlRenderTempEmulationSchema"),
  minCellCount = 0
)
\end{verbatim}
\end{Usage}
%
\begin{Arguments}
\begin{ldescription}
\item[\code{conceptIds}] An array of Concept ids.

\item[\code{connection}] An object of type \code{connection} as created using the
\code{\LinkA{connect}{connect}} function in the
DatabaseConnector package. Can be left NULL if \code{connectionDetails}
is provided, in which case a new connection will be opened at the start
of the function, and closed when the function finishes.

\item[\code{connectionDetails}] An object of type \code{connectionDetails} as created using the
\code{\LinkA{createConnectionDetails}{createConnectionDetails}} function in the
DatabaseConnector package. Can be left NULL if \code{connection} is
provided.

\item[\code{cdmDatabaseSchema}] Schema name where your patient-level data in OMOP CDM format resides.
Note that for SQL Server, this should include both the database and
schema name, for example 'cdm\_data.dbo'.

\item[\code{tempEmulationSchema}] Some database platforms like Oracle and Impala do not truly support temp tables. To emulate temp 
tables, provide a schema with write privileges where temp tables can be created.

\item[\code{minCellCount}] The minimum cell count for fields containing person/subject count.
\end{ldescription}
\end{Arguments}
%
\begin{Value}
Returns a tibble data frame.
\end{Value}
\inputencoding{utf8}
\HeaderA{getDomain}{Get all the domain id(s) in the vocabulary schema.}{getDomain}
%
\begin{Description}\relax
Get all the domain id(s) in the vocabulary schema.
\end{Description}
%
\begin{Usage}
\begin{verbatim}
getDomain(
  connection = NULL,
  connectionDetails = NULL,
  vocabularyDatabaseSchema = "vocabulary"
)
\end{verbatim}
\end{Usage}
%
\begin{Arguments}
\begin{ldescription}
\item[\code{connection}] An object of type \code{connection} as created using the
\code{\LinkA{connect}{connect}} function in the
DatabaseConnector package. Can be left NULL if \code{connectionDetails}
is provided, in which case a new connection will be opened at the start
of the function, and closed when the function finishes.

\item[\code{connectionDetails}] An object of type \code{connectionDetails} as created using the
\code{\LinkA{createConnectionDetails}{createConnectionDetails}} function in the
DatabaseConnector package. Can be left NULL if \code{connection} is
provided.

\item[\code{vocabularyDatabaseSchema}] The schema name of containing the vocabulary tables.
\end{ldescription}
\end{Arguments}
%
\begin{Value}
Returns a tibble data frame.
\end{Value}
\inputencoding{utf8}
\HeaderA{getDomainInformation}{Get domain information}{getDomainInformation}
%
\begin{Description}\relax
Get domain information
\end{Description}
%
\begin{Usage}
\begin{verbatim}
getDomainInformation(packageName = NULL)
\end{verbatim}
\end{Usage}
%
\begin{Arguments}
\begin{ldescription}
\item[\code{packageName}] e.g. 'CohortDiagnostics'
\end{ldescription}
\end{Arguments}
%
\begin{Value}
A list with two tibble data frame objects with domain information represented in wide and long format respectively.
\end{Value}
\inputencoding{utf8}
\HeaderA{getDrugIngredients}{Get ingredient information}{getDrugIngredients}
%
\begin{Description}\relax
Given an array of drug concept ids, returns their ingredients
\end{Description}
%
\begin{Usage}
\begin{verbatim}
getDrugIngredients(
  connection = NULL,
  connectionDetails = NULL,
  tempEmulationSchema = getOption("sqlRenderTempEmulationSchema"),
  conceptIds,
  vocabularyDatabaseSchema = "vocabulary"
)
\end{verbatim}
\end{Usage}
%
\begin{Arguments}
\begin{ldescription}
\item[\code{connection}] An object of type \code{connection} as created using the
\code{\LinkA{connect}{connect}} function in the
DatabaseConnector package. Can be left NULL if \code{connectionDetails}
is provided, in which case a new connection will be opened at the start
of the function, and closed when the function finishes.

\item[\code{connectionDetails}] An object of type \code{connectionDetails} as created using the
\code{\LinkA{createConnectionDetails}{createConnectionDetails}} function in the
DatabaseConnector package. Can be left NULL if \code{connection} is
provided.

\item[\code{tempEmulationSchema}] Some database platforms like Oracle and Impala do not truly support temp tables. To emulate temp 
tables, provide a schema with write privileges where temp tables can be created.

\item[\code{conceptIds}] An array of concept ids to find ingredients for

\item[\code{vocabularyDatabaseSchema}] The schema name of containing the vocabulary tables.
\end{ldescription}
\end{Arguments}
%
\begin{Value}
Returns a tibble data frame.
\end{Value}
\inputencoding{utf8}
\HeaderA{getExcludedConceptsInConceptSetExpression}{Given a concept set expression, get the resolved concepts}{getExcludedConceptsInConceptSetExpression}
%
\begin{Description}\relax
Given a concept set expression, get the resolved concepts
\end{Description}
%
\begin{Usage}
\begin{verbatim}
getExcludedConceptsInConceptSetExpression(
  conceptSetExpression,
  connection = NULL,
  connectionDetails = NULL,
  tempEmulationSchema = getOption("sqlRenderTempEmulationSchema"),
  vocabularyDatabaseSchema = "vocabulary"
)
\end{verbatim}
\end{Usage}
%
\begin{Arguments}
\begin{ldescription}
\item[\code{conceptSetExpression}] An R-object (list) with expression of the concept set.

\item[\code{connection}] An object of type \code{connection} as created using the
\code{\LinkA{connect}{connect}} function in the
DatabaseConnector package. Can be left NULL if \code{connectionDetails}
is provided, in which case a new connection will be opened at the start
of the function, and closed when the function finishes.

\item[\code{connectionDetails}] An object of type \code{connectionDetails} as created using the
\code{\LinkA{createConnectionDetails}{createConnectionDetails}} function in the
DatabaseConnector package. Can be left NULL if \code{connection} is
provided.

\item[\code{tempEmulationSchema}] Some database platforms like Oracle and Impala do not truly support temp tables. To emulate temp 
tables, provide a schema with write privileges where temp tables can be created.

\item[\code{vocabularyDatabaseSchema}] The schema name of containing the vocabulary tables.
\end{ldescription}
\end{Arguments}
%
\begin{Value}
Returns a tibble data frame.
\end{Value}
\inputencoding{utf8}
\HeaderA{getMappedSourceConcepts}{given a list of conceptIds, get their mapped}{getMappedSourceConcepts}
%
\begin{Description}\relax
Given a concept set expression, get the resolved concepts
\end{Description}
%
\begin{Usage}
\begin{verbatim}
getMappedSourceConcepts(
  conceptIds,
  connection = NULL,
  connectionDetails = NULL,
  tempEmulationSchema = getOption("sqlRenderTempEmulationSchema"),
  vocabularyDatabaseSchema = "vocabulary"
)
\end{verbatim}
\end{Usage}
%
\begin{Arguments}
\begin{ldescription}
\item[\code{conceptIds}] An array of Concept ids.

\item[\code{connection}] An object of type \code{connection} as created using the
\code{\LinkA{connect}{connect}} function in the
DatabaseConnector package. Can be left NULL if \code{connectionDetails}
is provided, in which case a new connection will be opened at the start
of the function, and closed when the function finishes.

\item[\code{connectionDetails}] An object of type \code{connectionDetails} as created using the
\code{\LinkA{createConnectionDetails}{createConnectionDetails}} function in the
DatabaseConnector package. Can be left NULL if \code{connection} is
provided.

\item[\code{tempEmulationSchema}] Some database platforms like Oracle and Impala do not truly support temp tables. To emulate temp 
tables, provide a schema with write privileges where temp tables can be created.

\item[\code{vocabularyDatabaseSchema}] The schema name of containing the vocabulary tables.
\end{ldescription}
\end{Arguments}
%
\begin{Value}
Returns a tibble data frame.
\end{Value}
\inputencoding{utf8}
\HeaderA{getMappedStandardConcepts}{given a list of conceptIds, get their mapped}{getMappedStandardConcepts}
%
\begin{Description}\relax
given a list of conceptIds, get their mapped
\end{Description}
%
\begin{Usage}
\begin{verbatim}
getMappedStandardConcepts(
  conceptIds,
  connection = NULL,
  connectionDetails = NULL,
  tempEmulationSchema = getOption("sqlRenderTempEmulationSchema"),
  vocabularyDatabaseSchema = "vocabulary"
)
\end{verbatim}
\end{Usage}
%
\begin{Arguments}
\begin{ldescription}
\item[\code{conceptIds}] An array of Concept ids.

\item[\code{connection}] An object of type \code{connection} as created using the
\code{\LinkA{connect}{connect}} function in the
DatabaseConnector package. Can be left NULL if \code{connectionDetails}
is provided, in which case a new connection will be opened at the start
of the function, and closed when the function finishes.

\item[\code{connectionDetails}] An object of type \code{connectionDetails} as created using the
\code{\LinkA{createConnectionDetails}{createConnectionDetails}} function in the
DatabaseConnector package. Can be left NULL if \code{connection} is
provided.

\item[\code{tempEmulationSchema}] Some database platforms like Oracle and Impala do not truly support temp tables. To emulate temp 
tables, provide a schema with write privileges where temp tables can be created.

\item[\code{vocabularyDatabaseSchema}] The schema name of containing the vocabulary tables.
\end{ldescription}
\end{Arguments}
%
\begin{Value}
Returns a tibble data frame.
\end{Value}
\inputencoding{utf8}
\HeaderA{getMedraRelationship}{get MedDRA relationship}{getMedraRelationship}
%
\begin{Description}\relax
given an array of conceptIds belonging to MedDRA vocabulary get its full MedDRA relationship
\end{Description}
%
\begin{Usage}
\begin{verbatim}
getMedraRelationship(
  conceptIds,
  connection = NULL,
  connectionDetails = NULL,
  tempEmulationSchema = getOption("sqlRenderTempEmulationSchema"),
  vocabularyDatabaseSchema = "vocabulary"
)
\end{verbatim}
\end{Usage}
%
\begin{Arguments}
\begin{ldescription}
\item[\code{conceptIds}] An array of Concept ids.

\item[\code{connection}] An object of type \code{connection} as created using the
\code{\LinkA{connect}{connect}} function in the
DatabaseConnector package. Can be left NULL if \code{connectionDetails}
is provided, in which case a new connection will be opened at the start
of the function, and closed when the function finishes.

\item[\code{connectionDetails}] An object of type \code{connectionDetails} as created using the
\code{\LinkA{createConnectionDetails}{createConnectionDetails}} function in the
DatabaseConnector package. Can be left NULL if \code{connection} is
provided.

\item[\code{tempEmulationSchema}] Some database platforms like Oracle and Impala do not truly support temp tables. To emulate temp 
tables, provide a schema with write privileges where temp tables can be created.

\item[\code{vocabularyDatabaseSchema}] The schema name of containing the vocabulary tables.
\end{ldescription}
\end{Arguments}
%
\begin{Value}
Returns a a list of tibble data frames
conceptId,
socConceptId, socConceptName,
HLTConceptId, HltConceptName,
HlgtConceptId, hlgtConceptName,
ptConceptId, ptConceptName,
lltConceptId, lltConceptName
\end{Value}
\inputencoding{utf8}
\HeaderA{getRecommendationForConceptSetExpression}{Get recommended concepts for a concept set expression.}{getRecommendationForConceptSetExpression}
%
\begin{Description}\relax
Get recommended concepts for a concept set expression.
\end{Description}
%
\begin{Usage}
\begin{verbatim}
getRecommendationForConceptSetExpression(
  conceptSetExpression,
  vocabularyDatabaseSchema = "vocabulary",
  connection = NULL,
  connectionDetails = NULL,
  conceptPrevalenceSchema = "concept_prevalence",
  tempEmulationSchema = getOption("sqlRenderTempEmulationSchema")
)
\end{verbatim}
\end{Usage}
%
\begin{Arguments}
\begin{ldescription}
\item[\code{vocabularyDatabaseSchema}] The schema name of containing the vocabulary tables.

\item[\code{connection}] An object of type \code{connection} as created using the
\code{\LinkA{connect}{connect}} function in the
DatabaseConnector package. Can be left NULL if \code{connectionDetails}
is provided, in which case a new connection will be opened at the start
of the function, and closed when the function finishes.

\item[\code{connectionDetails}] An object of type \code{connectionDetails} as created using the
\code{\LinkA{createConnectionDetails}{createConnectionDetails}} function in the
DatabaseConnector package. Can be left NULL if \code{connection} is
provided.

\item[\code{conceptPrevalenceSchema}] The schema name that has the concept prevalence table. The following
tables are expected to be present. recommender\_set,
cp\_master, recommended\_blacklist.

\item[\code{tempEmulationSchema}] Some database platforms like Oracle and Impala do not truly support temp tables. To emulate temp 
tables, provide a schema with write privileges where temp tables can be created.

\item[\code{conceptIds}] An array of Concept ids.
\end{ldescription}
\end{Arguments}
%
\begin{Value}
Returns a tibble data frame.
\end{Value}
\inputencoding{utf8}
\HeaderA{getRecommendedSource}{given a list of non standard conceptIds, get recommended conceptIds}{getRecommendedSource}
%
\begin{Description}\relax
given a list of non standard conceptIds, get recommended conceptIds
\end{Description}
%
\begin{Usage}
\begin{verbatim}
getRecommendedSource(
  conceptIds,
  vocabularyDatabaseSchema = "vocabulary",
  connection = NULL,
  connectionDetails = NULL,
  conceptPrevalenceSchema = "concept_prevalence",
  tempEmulationSchema = getOption("sqlRenderTempEmulationSchema")
)
\end{verbatim}
\end{Usage}
%
\begin{Arguments}
\begin{ldescription}
\item[\code{conceptIds}] An array of Concept ids.

\item[\code{vocabularyDatabaseSchema}] The schema name of containing the vocabulary tables.

\item[\code{connection}] An object of type \code{connection} as created using the
\code{\LinkA{connect}{connect}} function in the
DatabaseConnector package. Can be left NULL if \code{connectionDetails}
is provided, in which case a new connection will be opened at the start
of the function, and closed when the function finishes.

\item[\code{connectionDetails}] An object of type \code{connectionDetails} as created using the
\code{\LinkA{createConnectionDetails}{createConnectionDetails}} function in the
DatabaseConnector package. Can be left NULL if \code{connection} is
provided.

\item[\code{conceptPrevalenceSchema}] The schema name that has the concept prevalence table. The following
tables are expected to be present. recommender\_set,
cp\_master, recommended\_blacklist.

\item[\code{tempEmulationSchema}] Some database platforms like Oracle and Impala do not truly support temp tables. To emulate temp 
tables, provide a schema with write privileges where temp tables can be created.
\end{ldescription}
\end{Arguments}
\inputencoding{utf8}
\HeaderA{getRecommendedStandard}{given a list of standard conceptIds, get recommended concepts.}{getRecommendedStandard}
%
\begin{Description}\relax
given a list of standard conceptIds, get recommended concepts.
\end{Description}
%
\begin{Usage}
\begin{verbatim}
getRecommendedStandard(
  conceptIds,
  vocabularyDatabaseSchema,
  connection = NULL,
  connectionDetails = NULL,
  conceptPrevalenceSchema = "concept_prevalence",
  tempEmulationSchema = getOption("sqlRenderTempEmulationSchema")
)
\end{verbatim}
\end{Usage}
%
\begin{Arguments}
\begin{ldescription}
\item[\code{conceptIds}] An array of Concept ids.

\item[\code{vocabularyDatabaseSchema}] The schema name of containing the vocabulary tables.

\item[\code{connection}] An object of type \code{connection} as created using the
\code{\LinkA{connect}{connect}} function in the
DatabaseConnector package. Can be left NULL if \code{connectionDetails}
is provided, in which case a new connection will be opened at the start
of the function, and closed when the function finishes.

\item[\code{connectionDetails}] An object of type \code{connectionDetails} as created using the
\code{\LinkA{createConnectionDetails}{createConnectionDetails}} function in the
DatabaseConnector package. Can be left NULL if \code{connection} is
provided.

\item[\code{conceptPrevalenceSchema}] The schema name that has the concept prevalence table. The following
tables are expected to be present. recommender\_set,
cp\_master, recommended\_blacklist.

\item[\code{tempEmulationSchema}] Some database platforms like Oracle and Impala do not truly support temp tables. To emulate temp 
tables, provide a schema with write privileges where temp tables can be created.
\end{ldescription}
\end{Arguments}
\inputencoding{utf8}
\HeaderA{getRelationship}{get all relationship id from vocabulary tables in vocabulary schema.}{getRelationship}
%
\begin{Description}\relax
get all relationship id from vocabulary tables in vocabulary schema.
\end{Description}
%
\begin{Usage}
\begin{verbatim}
getRelationship(
  connection = NULL,
  connectionDetails = NULL,
  vocabularyDatabaseSchema = "vocabulary"
)
\end{verbatim}
\end{Usage}
%
\begin{Arguments}
\begin{ldescription}
\item[\code{connection}] An object of type \code{connection} as created using the
\code{\LinkA{connect}{connect}} function in the
DatabaseConnector package. Can be left NULL if \code{connectionDetails}
is provided, in which case a new connection will be opened at the start
of the function, and closed when the function finishes.

\item[\code{connectionDetails}] An object of type \code{connectionDetails} as created using the
\code{\LinkA{createConnectionDetails}{createConnectionDetails}} function in the
DatabaseConnector package. Can be left NULL if \code{connection} is
provided.

\item[\code{vocabularyDatabaseSchema}] The schema name of containing the vocabulary tables.
\end{ldescription}
\end{Arguments}
\inputencoding{utf8}
\HeaderA{getVocabulary}{Get vocabulary id(s) in vocabulary tables in vocabulary schema.}{getVocabulary}
%
\begin{Description}\relax
Get vocabulary id(s) in vocabulary tables in vocabulary schema.
\end{Description}
%
\begin{Usage}
\begin{verbatim}
getVocabulary(
  connection = NULL,
  connectionDetails = NULL,
  vocabularyDatabaseSchema = "vocabulary"
)
\end{verbatim}
\end{Usage}
%
\begin{Arguments}
\begin{ldescription}
\item[\code{connection}] An object of type \code{connection} as created using the
\code{\LinkA{connect}{connect}} function in the
DatabaseConnector package. Can be left NULL if \code{connectionDetails}
is provided, in which case a new connection will be opened at the start
of the function, and closed when the function finishes.

\item[\code{connectionDetails}] An object of type \code{connectionDetails} as created using the
\code{\LinkA{createConnectionDetails}{createConnectionDetails}} function in the
DatabaseConnector package. Can be left NULL if \code{connection} is
provided.

\item[\code{vocabularyDatabaseSchema}] The schema name of containing the vocabulary tables.
\end{ldescription}
\end{Arguments}
\inputencoding{utf8}
\HeaderA{getVocabularyVersion}{Get vocabulary version.}{getVocabularyVersion}
%
\begin{Description}\relax
Get vocabulary version.
\end{Description}
%
\begin{Usage}
\begin{verbatim}
getVocabularyVersion(
  connection = NULL,
  connectionDetails = NULL,
  vocabularyDatabaseSchema = "vocabulary"
)
\end{verbatim}
\end{Usage}
%
\begin{Arguments}
\begin{ldescription}
\item[\code{connection}] An object of type \code{connection} as created using the
\code{\LinkA{connect}{connect}} function in the
DatabaseConnector package. Can be left NULL if \code{connectionDetails}
is provided, in which case a new connection will be opened at the start
of the function, and closed when the function finishes.

\item[\code{connectionDetails}] An object of type \code{connectionDetails} as created using the
\code{\LinkA{createConnectionDetails}{createConnectionDetails}} function in the
DatabaseConnector package. Can be left NULL if \code{connection} is
provided.

\item[\code{vocabularyDatabaseSchema}] The schema name of containing the vocabulary tables.
\end{ldescription}
\end{Arguments}
\inputencoding{utf8}
\HeaderA{mapMedraToSnomedViaVocabulary}{map MedDRA to SNOMED}{mapMedraToSnomedViaVocabulary}
%
\begin{Description}\relax
given an array of conceptIds belonging to MedDRA vocabulary get its equivalent SNOMED ranked
using a combination of OMOP vocabulary mapping, lexical string matching and concept prevalence
counts
\end{Description}
%
\begin{Usage}
\begin{verbatim}
mapMedraToSnomedViaVocabulary(
  conceptIds,
  connection = NULL,
  connectionDetails = NULL,
  tempEmulationSchema = getOption("sqlRenderTempEmulationSchema"),
  vocabularyDatabaseSchema = "vocabulary"
)
\end{verbatim}
\end{Usage}
%
\begin{Arguments}
\begin{ldescription}
\item[\code{conceptIds}] An array of Concept ids.

\item[\code{connection}] An object of type \code{connection} as created using the
\code{\LinkA{connect}{connect}} function in the
DatabaseConnector package. Can be left NULL if \code{connectionDetails}
is provided, in which case a new connection will be opened at the start
of the function, and closed when the function finishes.

\item[\code{connectionDetails}] An object of type \code{connectionDetails} as created using the
\code{\LinkA{createConnectionDetails}{createConnectionDetails}} function in the
DatabaseConnector package. Can be left NULL if \code{connection} is
provided.

\item[\code{tempEmulationSchema}] Some database platforms like Oracle and Impala do not truly support temp tables. To emulate temp 
tables, provide a schema with write privileges where temp tables can be created.

\item[\code{vocabularyDatabaseSchema}] The schema name of containing the vocabulary tables.
\end{ldescription}
\end{Arguments}
%
\begin{Value}
Returns a tibble data frame
\end{Value}
\inputencoding{utf8}
\HeaderA{optimizeConceptSetExpression}{given a concept set expression, get optimized concept set expression}{optimizeConceptSetExpression}
%
\begin{Description}\relax
given a concept set expression, get optimized concept set expression
\end{Description}
%
\begin{Usage}
\begin{verbatim}
optimizeConceptSetExpression(
  conceptSetExpression,
  vocabularyDatabaseSchema = "vocabulary",
  connection = NULL,
  tempEmulationSchema = getOption("sqlRenderTempEmulationSchema"),
  connectionDetails = NULL
)
\end{verbatim}
\end{Usage}
%
\begin{Arguments}
\begin{ldescription}
\item[\code{conceptSetExpression}] An R-object (list) with expression of the concept set.

\item[\code{vocabularyDatabaseSchema}] The schema name of containing the vocabulary tables.

\item[\code{connection}] An object of type \code{connection} as created using the
\code{\LinkA{connect}{connect}} function in the
DatabaseConnector package. Can be left NULL if \code{connectionDetails}
is provided, in which case a new connection will be opened at the start
of the function, and closed when the function finishes.

\item[\code{tempEmulationSchema}] Some database platforms like Oracle and Impala do not truly support temp tables. To emulate temp 
tables, provide a schema with write privileges where temp tables can be created.

\item[\code{connectionDetails}] An object of type \code{connectionDetails} as created using the
\code{\LinkA{createConnectionDetails}{createConnectionDetails}} function in the
DatabaseConnector package. Can be left NULL if \code{connection} is
provided.
\end{ldescription}
\end{Arguments}
\inputencoding{utf8}
\HeaderA{performStringSearchForConcepts}{Get concepts that match a string search}{performStringSearchForConcepts}
%
\begin{Description}\relax
Get concepts that match a string search
\end{Description}
%
\begin{Usage}
\begin{verbatim}
performStringSearchForConcepts(
  searchString,
  vocabularyDatabaseSchema = "vocabulary",
  connection = NULL,
  connectionDetails = NULL
)
\end{verbatim}
\end{Usage}
%
\begin{Arguments}
\begin{ldescription}
\item[\code{searchString}] A phrase (can be multiple words) to search for.

\item[\code{vocabularyDatabaseSchema}] The schema name of containing the vocabulary tables.

\item[\code{connection}] An object of type \code{connection} as created using the
\code{\LinkA{connect}{connect}} function in the
DatabaseConnector package. Can be left NULL if \code{connectionDetails}
is provided, in which case a new connection will be opened at the start
of the function, and closed when the function finishes.

\item[\code{connectionDetails}] An object of type \code{connectionDetails} as created using the
\code{\LinkA{createConnectionDetails}{createConnectionDetails}} function in the
DatabaseConnector package. Can be left NULL if \code{connection} is
provided.
\end{ldescription}
\end{Arguments}
\inputencoding{utf8}
\HeaderA{resolveConceptSetExpression}{Given a concept set expression, get the resolved concepts}{resolveConceptSetExpression}
%
\begin{Description}\relax
Given a concept set expression, get the resolved concepts
\end{Description}
%
\begin{Usage}
\begin{verbatim}
resolveConceptSetExpression(
  conceptSetExpression,
  connection = NULL,
  connectionDetails = NULL,
  vocabularyDatabaseSchema = "vocabulary"
)
\end{verbatim}
\end{Usage}
%
\begin{Arguments}
\begin{ldescription}
\item[\code{conceptSetExpression}] An R-object (list) with expression of the concept set.

\item[\code{connection}] An object of type \code{connection} as created using the
\code{\LinkA{connect}{connect}} function in the
DatabaseConnector package. Can be left NULL if \code{connectionDetails}
is provided, in which case a new connection will be opened at the start
of the function, and closed when the function finishes.

\item[\code{connectionDetails}] An object of type \code{connectionDetails} as created using the
\code{\LinkA{createConnectionDetails}{createConnectionDetails}} function in the
DatabaseConnector package. Can be left NULL if \code{connection} is
provided.

\item[\code{vocabularyDatabaseSchema}] The schema name of containing the vocabulary tables.
\end{ldescription}
\end{Arguments}
%
\begin{Value}
Returns a tibble data frame.
\end{Value}
\inputencoding{utf8}
\HeaderA{resolveConceptSetsInCohortExpression}{Given a cohort definition expression, get the resolved concepts for all concept sets}{resolveConceptSetsInCohortExpression}
%
\begin{Description}\relax
Given a cohort definition expression, get the resolved concepts for all concept sets
\end{Description}
%
\begin{Usage}
\begin{verbatim}
resolveConceptSetsInCohortExpression(
  cohortExpression,
  connection = NULL,
  connectionDetails = NULL,
  vocabularyDatabaseSchema = "vocabulary"
)
\end{verbatim}
\end{Usage}
%
\begin{Arguments}
\begin{ldescription}
\item[\code{cohortExpression}] A R-object (list) that represents cohort definition expression. This is derived from cohort expression json 
using RJSONIO::fromJSON(content = json, digits = 23). Note: it is important to use digits = 23, otherwise
numerical precision may be lost for large integer values like conceptId's in cohort definition. The cohort
expression JSON is commonly generated using OHDSI tools like Atlas or CapR.

\item[\code{connection}] An object of type \code{connection} as created using the
\code{\LinkA{connect}{connect}} function in the
DatabaseConnector package. Can be left NULL if \code{connectionDetails}
is provided, in which case a new connection will be opened at the start
of the function, and closed when the function finishes.

\item[\code{connectionDetails}] An object of type \code{connectionDetails} as created using the
\code{\LinkA{createConnectionDetails}{createConnectionDetails}} function in the
DatabaseConnector package. Can be left NULL if \code{connection} is
provided.

\item[\code{vocabularyDatabaseSchema}] The schema name of containing the vocabulary tables.
\end{ldescription}
\end{Arguments}
\printindex{}
\end{document}
